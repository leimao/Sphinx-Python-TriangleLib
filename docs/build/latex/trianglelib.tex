%% Generated by Sphinx.
\def\sphinxdocclass{report}
\documentclass[letterpaper,10pt,english]{sphinxmanual}
\ifdefined\pdfpxdimen
   \let\sphinxpxdimen\pdfpxdimen\else\newdimen\sphinxpxdimen
\fi \sphinxpxdimen=.75bp\relax

\PassOptionsToPackage{warn}{textcomp}
\usepackage[utf8]{inputenc}
\ifdefined\DeclareUnicodeCharacter
% support both utf8 and utf8x syntaxes
  \ifdefined\DeclareUnicodeCharacterAsOptional
    \def\sphinxDUC#1{\DeclareUnicodeCharacter{"#1}}
  \else
    \let\sphinxDUC\DeclareUnicodeCharacter
  \fi
  \sphinxDUC{00A0}{\nobreakspace}
  \sphinxDUC{2500}{\sphinxunichar{2500}}
  \sphinxDUC{2502}{\sphinxunichar{2502}}
  \sphinxDUC{2514}{\sphinxunichar{2514}}
  \sphinxDUC{251C}{\sphinxunichar{251C}}
  \sphinxDUC{2572}{\textbackslash}
\fi
\usepackage{cmap}
\usepackage[T1]{fontenc}
\usepackage{amsmath,amssymb,amstext}
\usepackage{babel}



\usepackage{times}
\expandafter\ifx\csname T@LGR\endcsname\relax
\else
% LGR was declared as font encoding
  \substitutefont{LGR}{\rmdefault}{cmr}
  \substitutefont{LGR}{\sfdefault}{cmss}
  \substitutefont{LGR}{\ttdefault}{cmtt}
\fi
\expandafter\ifx\csname T@X2\endcsname\relax
  \expandafter\ifx\csname T@T2A\endcsname\relax
  \else
  % T2A was declared as font encoding
    \substitutefont{T2A}{\rmdefault}{cmr}
    \substitutefont{T2A}{\sfdefault}{cmss}
    \substitutefont{T2A}{\ttdefault}{cmtt}
  \fi
\else
% X2 was declared as font encoding
  \substitutefont{X2}{\rmdefault}{cmr}
  \substitutefont{X2}{\sfdefault}{cmss}
  \substitutefont{X2}{\ttdefault}{cmtt}
\fi


\usepackage[Bjarne]{fncychap}
\usepackage{sphinx}

\fvset{fontsize=\small}
\usepackage{geometry}


% Include hyperref last.
\usepackage{hyperref}
% Fix anchor placement for figures with captions.
\usepackage{hypcap}% it must be loaded after hyperref.
% Set up styles of URL: it should be placed after hyperref.
\urlstyle{same}

\addto\captionsenglish{\renewcommand{\contentsname}{Contents:}}

\usepackage{sphinxmessages}
\setcounter{tocdepth}{1}



\title{trianglelib}
\date{Aug 01, 2020}
\release{1.1}
\author{Brandon Rhodes, Lei Mao}
\newcommand{\sphinxlogo}{\vbox{}}
\renewcommand{\releasename}{Release}
\makeindex
\begin{document}

\pagestyle{empty}
\sphinxmaketitle
\pagestyle{plain}
\sphinxtableofcontents
\pagestyle{normal}
\phantomsection\label{\detokenize{index::doc}}



\chapter{Example: guide.rst — The trianglelib guide}
\label{\detokenize{guide:example-guide-rst-the-trianglelib-guide}}\label{\detokenize{guide::doc}}
Whether you need to test the properties of triangles,
or learn their dimensions, {\hyperref[\detokenize{api:module-trianglelib}]{\sphinxcrossref{\sphinxcode{\sphinxupquote{trianglelib}}}}} does it all!


\section{Special triangles}
\label{\detokenize{guide:special-triangles}}
There are two special kinds of triangle
for which {\hyperref[\detokenize{api:module-trianglelib}]{\sphinxcrossref{\sphinxcode{\sphinxupquote{trianglelib}}}}} offers special support.
\begin{description}
\item[{\sphinxstyleemphasis{Equilateral triangle}}] \leavevmode
All three sides are of equal length.

\item[{\sphinxstyleemphasis{Isosceles triangle}}] \leavevmode
Has at least two sides that are of equal length.

\end{description}

These are supported both by simple methods
that are available in the {\hyperref[\detokenize{api:module-trianglelib.utils}]{\sphinxcrossref{\sphinxcode{\sphinxupquote{trianglelib.utils}}}}} module,
and also by a pair of methods of the main
{\hyperref[\detokenize{api:trianglelib.shape.Triangle}]{\sphinxcrossref{\sphinxcode{\sphinxupquote{Triangle}}}}} class itself.


\section{Triangle dimensions}
\label{\detokenize{guide:triangle-dimensions}}\label{\detokenize{guide:id1}}
The library can compute triangle perimeter, area,
and can also compare two triangles for equality.
Note that it does not matter which side you start with,
so long as two triangles have the same three sides in the same order!

\begin{sphinxVerbatim}[commandchars=\\\{\}]
\PYG{g+gp}{\PYGZgt{}\PYGZgt{}\PYGZgt{} }\PYG{k+kn}{from} \PYG{n+nn}{trianglelib}\PYG{n+nn}{.}\PYG{n+nn}{shape} \PYG{k+kn}{import} \PYG{n}{Triangle}
\PYG{g+gp}{\PYGZgt{}\PYGZgt{}\PYGZgt{} }\PYG{n}{t1} \PYG{o}{=} \PYG{n}{Triangle}\PYG{p}{(}\PYG{l+m+mi}{3}\PYG{p}{,} \PYG{l+m+mi}{4}\PYG{p}{,} \PYG{l+m+mi}{5}\PYG{p}{)}
\PYG{g+gp}{\PYGZgt{}\PYGZgt{}\PYGZgt{} }\PYG{n}{t2} \PYG{o}{=} \PYG{n}{Triangle}\PYG{p}{(}\PYG{l+m+mi}{4}\PYG{p}{,} \PYG{l+m+mi}{5}\PYG{p}{,} \PYG{l+m+mi}{3}\PYG{p}{)}
\PYG{g+gp}{\PYGZgt{}\PYGZgt{}\PYGZgt{} }\PYG{n}{t3} \PYG{o}{=} \PYG{n}{Triangle}\PYG{p}{(}\PYG{l+m+mi}{3}\PYG{p}{,} \PYG{l+m+mi}{4}\PYG{p}{,} \PYG{l+m+mi}{6}\PYG{p}{)}
\PYG{g+gp}{\PYGZgt{}\PYGZgt{}\PYGZgt{} }\PYG{n+nb}{print}\PYG{p}{(}\PYG{n}{t1} \PYG{o}{==} \PYG{n}{t2}\PYG{p}{)}
\PYG{g+go}{True}
\PYG{g+gp}{\PYGZgt{}\PYGZgt{}\PYGZgt{} }\PYG{n+nb}{print}\PYG{p}{(}\PYG{n}{t1} \PYG{o}{==} \PYG{n}{t3}\PYG{p}{)}
\PYG{g+go}{False}
\PYG{g+gp}{\PYGZgt{}\PYGZgt{}\PYGZgt{} }\PYG{n+nb}{print}\PYG{p}{(}\PYG{n}{t1}\PYG{o}{.}\PYG{n}{area}\PYG{p}{(}\PYG{p}{)}\PYG{p}{)}
\PYG{g+go}{6.0}
\PYG{g+gp}{\PYGZgt{}\PYGZgt{}\PYGZgt{} }\PYG{n+nb}{print}\PYG{p}{(}\PYG{n}{t1}\PYG{o}{.}\PYG{n}{scale}\PYG{p}{(}\PYG{l+m+mf}{2.0}\PYG{p}{)}\PYG{o}{.}\PYG{n}{area}\PYG{p}{(}\PYG{p}{)}\PYG{p}{)}
\PYG{g+go}{24.0}
\end{sphinxVerbatim}


\section{Valid triangles}
\label{\detokenize{guide:valid-triangles}}
Many combinations of three numbers cannot be the sides of a triangle.
Even if all three numbers are positive instead of negative or zero,
one of the numbers can still be so large
that the shorter two sides
could not actually meet to make a closed figure.
If \(c\) is the longest side, then a triangle is only possible if:
\begin{equation*}
\begin{split}a + b > c\end{split}
\end{equation*}
While the documentation
for each function in the {\hyperref[\detokenize{api:module-trianglelib.utils}]{\sphinxcrossref{\sphinxcode{\sphinxupquote{utils}}}}} module
simply specifies a return value for cases that are not real triangles,
the {\hyperref[\detokenize{api:trianglelib.shape.Triangle}]{\sphinxcrossref{\sphinxcode{\sphinxupquote{Triangle}}}}} class is more strict
and raises an exception if your sides lengths are not appropriate:

\begin{sphinxVerbatim}[commandchars=\\\{\}]
\PYG{g+gp}{\PYGZgt{}\PYGZgt{}\PYGZgt{} }\PYG{k+kn}{from} \PYG{n+nn}{trianglelib}\PYG{n+nn}{.}\PYG{n+nn}{shape} \PYG{k+kn}{import} \PYG{n}{Triangle}
\PYG{g+gp}{\PYGZgt{}\PYGZgt{}\PYGZgt{} }\PYG{n}{Triangle}\PYG{p}{(}\PYG{l+m+mi}{1}\PYG{p}{,} \PYG{l+m+mi}{1}\PYG{p}{,} \PYG{l+m+mi}{3}\PYG{p}{)}
\PYG{g+gt}{Traceback (most recent call last):}
    \PYG{o}{.}\PYG{o}{.}\PYG{o}{.}
\PYG{g+gr}{ValueError}: \PYG{n}{one side is too long to make a triangle}
\end{sphinxVerbatim}

If you are not sanitizing your user input
to verify that the three side lengths they are giving you are safe,
then be prepared to trap this exception
and report the error to your user.


\chapter{Example: tutorial.rst — The trianglelib tutorial}
\label{\detokenize{tutorial:example-tutorial-rst-the-trianglelib-tutorial}}\label{\detokenize{tutorial::doc}}\begin{quote}

\index{Euclid@\spxentry{Euclid}}\ignorespaces 
\sphinxstyleemphasis{“There is no royal road to geometry.”} — Euclid
\end{quote}

This module makes triangle processing fun!
The beginner will enjoy how the {\hyperref[\detokenize{api:module-trianglelib.utils}]{\sphinxcrossref{\sphinxcode{\sphinxupquote{utils}}}}} module
lets you get started quickly.

\begin{sphinxVerbatim}[commandchars=\\\{\}]
\PYG{g+gp}{\PYGZgt{}\PYGZgt{}\PYGZgt{} }\PYG{k+kn}{from} \PYG{n+nn}{trianglelib} \PYG{k+kn}{import} \PYG{n}{utils}
\PYG{g+gp}{\PYGZgt{}\PYGZgt{}\PYGZgt{} }\PYG{n}{utils}\PYG{o}{.}\PYG{n}{is\PYGZus{}isosceles}\PYG{p}{(}\PYG{l+m+mi}{5}\PYG{p}{,} \PYG{l+m+mi}{5}\PYG{p}{,} \PYG{l+m+mi}{7}\PYG{p}{)}
\PYG{g+go}{True}
\end{sphinxVerbatim}

But fancier programmers can use the {\hyperref[\detokenize{api:trianglelib.shape.Triangle}]{\sphinxcrossref{\sphinxcode{\sphinxupquote{Triangle}}}}}
class to create an actual triangle \sphinxstyleemphasis{object}
upon which they can then perform lots of operations.
For example, consider this Python program:

\begin{sphinxVerbatim}[commandchars=\\\{\}]
\PYG{k+kn}{from} \PYG{n+nn}{trianglelib}\PYG{n+nn}{.}\PYG{n+nn}{shape} \PYG{k+kn}{import} \PYG{n}{Triangle}
\PYG{n}{t} \PYG{o}{=} \PYG{n}{Triangle}\PYG{p}{(}\PYG{l+m+mi}{5}\PYG{p}{,} \PYG{l+m+mi}{5}\PYG{p}{,} \PYG{l+m+mi}{5}\PYG{p}{)}
\PYG{n+nb}{print}\PYG{p}{(}\PYG{l+s+s1}{\PYGZsq{}}\PYG{l+s+s1}{Equilateral?}\PYG{l+s+s1}{\PYGZsq{}}\PYG{p}{,} \PYG{n}{t}\PYG{o}{.}\PYG{n}{is\PYGZus{}equilateral}\PYG{p}{(}\PYG{p}{)}\PYG{p}{)}
\PYG{n+nb}{print}\PYG{p}{(}\PYG{l+s+s1}{\PYGZsq{}}\PYG{l+s+s1}{Isosceles?}\PYG{l+s+s1}{\PYGZsq{}}\PYG{p}{,} \PYG{n}{t}\PYG{o}{.}\PYG{n}{is\PYGZus{}isosceles}\PYG{p}{(}\PYG{p}{)}\PYG{p}{)}
\end{sphinxVerbatim}

Since methods like {\hyperref[\detokenize{api:trianglelib.shape.Triangle.is_equilateral}]{\sphinxcrossref{\sphinxcode{\sphinxupquote{is\_equilateral()}}}}}
return Boolean values, this program will produce the following output:

\begin{sphinxVerbatim}[commandchars=\\\{\}]
Equilateral? True
Isosceles? True
\end{sphinxVerbatim}

Read {\hyperref[\detokenize{guide::doc}]{\sphinxcrossref{\DUrole{doc}{Example: guide.rst — The trianglelib guide}}}} to learn more!

\begin{sphinxadmonition}{warning}{Warning:}
This module only handles three\sphinxhyphen{}sided polygons;
five\sphinxhyphen{}sided figures are right out.
\end{sphinxadmonition}


\chapter{The trianglelib API Reference}
\label{\detokenize{api:module-trianglelib}}\label{\detokenize{api:the-trianglelib-api-reference}}\label{\detokenize{api::doc}}\index{module@\spxentry{module}!trianglelib@\spxentry{trianglelib}}\index{trianglelib@\spxentry{trianglelib}!module@\spxentry{module}}
Routines for working with triangles.

The two modules inside of this package are packed with useful features
for the programmer who needs to support triangles:
\begin{description}
\item[{\sphinxcode{\sphinxupquote{shape}}}] \leavevmode
This module provides a full\sphinxhyphen{}fledged \sphinxtitleref{Triangle} object that can be
instantiated and then asked to provide all sorts of information
about its properties.

\item[{\sphinxcode{\sphinxupquote{utils}}}] \leavevmode
For the programmer in a hurry, this module offers quick functions
that take as arguments the three side lengths of a triangle, and
perform a quick computation without the programmer having to make
the extra step of creating an object.

\end{description}


\section{The “shape” module}
\label{\detokenize{api:module-trianglelib.shape}}\label{\detokenize{api:the-shape-module}}\index{module@\spxentry{module}!trianglelib.shape@\spxentry{trianglelib.shape}}\index{trianglelib.shape@\spxentry{trianglelib.shape}!module@\spxentry{module}}
Use the triangle class to represent triangles.
\index{Triangle (class in trianglelib.shape)@\spxentry{Triangle}\spxextra{class in trianglelib.shape}}

\begin{fulllineitems}
\phantomsection\label{\detokenize{api:trianglelib.shape.Triangle}}\pysiglinewithargsret{\sphinxbfcode{\sphinxupquote{class }}\sphinxcode{\sphinxupquote{trianglelib.shape.}}\sphinxbfcode{\sphinxupquote{Triangle}}}{\emph{\DUrole{n}{a}}, \emph{\DUrole{n}{b}}, \emph{\DUrole{n}{c}}}{}
A {\hyperref[\detokenize{api:trianglelib.shape.Triangle}]{\sphinxcrossref{\sphinxcode{\sphinxupquote{Triangle}}}}} object is a three\sphinxhyphen{}sided polygon.

You instantiate a {\hyperref[\detokenize{api:trianglelib.shape.Triangle}]{\sphinxcrossref{\sphinxcode{\sphinxupquote{Triangle}}}}} by providing exactly three lengths \sphinxcode{\sphinxupquote{a}}, \sphinxcode{\sphinxupquote{b}}, and \sphinxcode{\sphinxupquote{c}}.

They can either be intergers or floating\sphinxhyphen{}point numbers, and should be listed clockwise around the triangle.

If the three lengths \sphinxstyleemphasis{cannot} make a valid triangle, then \sphinxcode{\sphinxupquote{ValueError}} will be raised instead.

\begin{sphinxVerbatim}[commandchars=\\\{\}]
\PYG{g+gp}{\PYGZgt{}\PYGZgt{}\PYGZgt{} }\PYG{k+kn}{from} \PYG{n+nn}{trianglelib}\PYG{n+nn}{.}\PYG{n+nn}{shape} \PYG{k+kn}{import} \PYG{n}{Triangle}
\PYG{g+gp}{\PYGZgt{}\PYGZgt{}\PYGZgt{} }\PYG{n}{t} \PYG{o}{=} \PYG{n}{Triangle}\PYG{p}{(}\PYG{l+m+mi}{3}\PYG{p}{,} \PYG{l+m+mi}{4}\PYG{p}{,} \PYG{l+m+mi}{5}\PYG{p}{)}
\PYG{g+gp}{\PYGZgt{}\PYGZgt{}\PYGZgt{} }\PYG{n+nb}{print}\PYG{p}{(}\PYG{n}{t}\PYG{o}{.}\PYG{n}{is\PYGZus{}equilateral}\PYG{p}{(}\PYG{p}{)}\PYG{p}{)}
\PYG{g+go}{False}
\PYG{g+gp}{\PYGZgt{}\PYGZgt{}\PYGZgt{} }\PYG{n+nb}{print}\PYG{p}{(}\PYG{n}{t}\PYG{o}{.}\PYG{n}{area}\PYG{p}{(}\PYG{p}{)}\PYG{p}{)}
\PYG{g+go}{6.0}
\end{sphinxVerbatim}

Triangles support the following attributes, operators, and methods.
\index{a (trianglelib.shape.Triangle attribute)@\spxentry{a}\spxextra{trianglelib.shape.Triangle attribute}}\index{b (trianglelib.shape.Triangle attribute)@\spxentry{b}\spxextra{trianglelib.shape.Triangle attribute}}\index{c (trianglelib.shape.Triangle attribute)@\spxentry{c}\spxextra{trianglelib.shape.Triangle attribute}}

\begin{fulllineitems}
\phantomsection\label{\detokenize{api:trianglelib.shape.Triangle.a}}\pysigline{\sphinxbfcode{\sphinxupquote{a}}}\phantomsection\label{\detokenize{api:trianglelib.shape.Triangle.b}}\pysigline{\sphinxbfcode{\sphinxupquote{b}}}\phantomsection\label{\detokenize{api:trianglelib.shape.Triangle.c}}\pysigline{\sphinxbfcode{\sphinxupquote{c}}}
The three side lengths provided during instantiation.

\end{fulllineitems}


\index{equality@\spxentry{equality}!triangle@\spxentry{triangle}}\index{triangle@\spxentry{triangle}!equality@\spxentry{equality}}\ignorespaces 

\begin{fulllineitems}
\pysigline{\sphinxbfcode{\sphinxupquote{triangle1~==~triangle2}}}
Returns true if the two triangles have sides of the same lengths,
in the same order.
Note that it is okay if the two triangles
happen to start their list of sides at a different corner;
\sphinxcode{\sphinxupquote{3,4,5}} is the same triangle as \sphinxcode{\sphinxupquote{4,5,3}}
but neither of these are the same triangle
as their mirror image \sphinxcode{\sphinxupquote{5,4,3}}.

\end{fulllineitems}

\index{\_\_init\_\_() (trianglelib.shape.Triangle method)@\spxentry{\_\_init\_\_()}\spxextra{trianglelib.shape.Triangle method}}

\begin{fulllineitems}
\phantomsection\label{\detokenize{api:trianglelib.shape.Triangle.__init__}}\pysiglinewithargsret{\sphinxbfcode{\sphinxupquote{\_\_init\_\_}}}{\emph{\DUrole{n}{a}}, \emph{\DUrole{n}{b}}, \emph{\DUrole{n}{c}}}{}
Create a {\hyperref[\detokenize{api:trianglelib.shape.Triangle}]{\sphinxcrossref{\sphinxcode{\sphinxupquote{Triangle}}}}} object with sides of lengths \sphinxtitleref{a}, \sphinxtitleref{b}, and \sphinxtitleref{c}.

Raises \sphinxtitleref{ValueError} if the three length values provided cannot
actually form a triangle.
\begin{quote}\begin{description}
\item[{Parameters}] \leavevmode\begin{itemize}
\item {} 
\sphinxstyleliteralstrong{\sphinxupquote{a}} (\sphinxcode{\sphinxupquote{float}}) \textendash{} side length one

\item {} 
\sphinxstyleliteralstrong{\sphinxupquote{b}} (\sphinxcode{\sphinxupquote{float}}) \textendash{} side length two

\item {} 
\sphinxstyleliteralstrong{\sphinxupquote{c}} (\sphinxcode{\sphinxupquote{float}}) \textendash{} side length three

\end{itemize}

\item[{Raises}] \leavevmode\begin{itemize}
\item {} 
\sphinxstyleliteralstrong{\sphinxupquote{ValueError}} \textendash{} side lengths must all be positive

\item {} 
\sphinxstyleliteralstrong{\sphinxupquote{ValueError}} \textendash{} one side is too long to make a triangle

\end{itemize}

\end{description}\end{quote}

\end{fulllineitems}

\index{is\_equivalent() (trianglelib.shape.Triangle method)@\spxentry{is\_equivalent()}\spxextra{trianglelib.shape.Triangle method}}

\begin{fulllineitems}
\phantomsection\label{\detokenize{api:trianglelib.shape.Triangle.is_equivalent}}\pysiglinewithargsret{\sphinxbfcode{\sphinxupquote{is\_equivalent}}}{\emph{\DUrole{n}{triangle}}}{}
Return whether this triangle equals another triangle.
\begin{quote}\begin{description}
\item[{Parameters}] \leavevmode
\sphinxstyleliteralstrong{\sphinxupquote{triangle}} ({\hyperref[\detokenize{api:trianglelib.shape.Triangle}]{\sphinxcrossref{\sphinxcode{\sphinxupquote{Triangle}}}}}) \textendash{} another {\hyperref[\detokenize{api:trianglelib.shape.Triangle}]{\sphinxcrossref{\sphinxcode{\sphinxupquote{Triangle}}}}} object

\item[{Returns}] \leavevmode
whether the two {\hyperref[\detokenize{api:trianglelib.shape.Triangle}]{\sphinxcrossref{\sphinxcode{\sphinxupquote{Triangle}}}}} objects are equivalent

\item[{Return type}] \leavevmode
\sphinxcode{\sphinxupquote{bool}}

\end{description}\end{quote}

\end{fulllineitems}

\index{is\_similar() (trianglelib.shape.Triangle method)@\spxentry{is\_similar()}\spxextra{trianglelib.shape.Triangle method}}

\begin{fulllineitems}
\phantomsection\label{\detokenize{api:trianglelib.shape.Triangle.is_similar}}\pysiglinewithargsret{\sphinxbfcode{\sphinxupquote{is\_similar}}}{\emph{\DUrole{n}{triangle}}}{}
Return whether this {\hyperref[\detokenize{api:trianglelib.shape.Triangle}]{\sphinxcrossref{\sphinxcode{\sphinxupquote{Triangle}}}}} object is similar to another triangle.
\begin{quote}\begin{description}
\item[{Parameters}] \leavevmode
\sphinxstyleliteralstrong{\sphinxupquote{triangle}} ({\hyperref[\detokenize{api:trianglelib.shape.Triangle}]{\sphinxcrossref{\sphinxcode{\sphinxupquote{Triangle}}}}}) \textendash{} another {\hyperref[\detokenize{api:trianglelib.shape.Triangle}]{\sphinxcrossref{\sphinxcode{\sphinxupquote{Triangle}}}}} object

\item[{Returns}] \leavevmode
whether the two {\hyperref[\detokenize{api:trianglelib.shape.Triangle}]{\sphinxcrossref{\sphinxcode{\sphinxupquote{Triangle}}}}} objects are similar

\item[{Return type}] \leavevmode
\sphinxcode{\sphinxupquote{bool}}

\end{description}\end{quote}

\end{fulllineitems}

\index{is\_equilateral() (trianglelib.shape.Triangle method)@\spxentry{is\_equilateral()}\spxextra{trianglelib.shape.Triangle method}}

\begin{fulllineitems}
\phantomsection\label{\detokenize{api:trianglelib.shape.Triangle.is_equilateral}}\pysiglinewithargsret{\sphinxbfcode{\sphinxupquote{is\_equilateral}}}{}{}
Return whether this {\hyperref[\detokenize{api:trianglelib.shape.Triangle}]{\sphinxcrossref{\sphinxcode{\sphinxupquote{Triangle}}}}} object is equilateral.
\begin{quote}\begin{description}
\item[{Returns}] \leavevmode
whether the {\hyperref[\detokenize{api:trianglelib.shape.Triangle}]{\sphinxcrossref{\sphinxcode{\sphinxupquote{Triangle}}}}} object is equilateral

\item[{Return type}] \leavevmode
\sphinxcode{\sphinxupquote{bool}}

\end{description}\end{quote}

\end{fulllineitems}

\index{is\_isosceles() (trianglelib.shape.Triangle method)@\spxentry{is\_isosceles()}\spxextra{trianglelib.shape.Triangle method}}

\begin{fulllineitems}
\phantomsection\label{\detokenize{api:trianglelib.shape.Triangle.is_isosceles}}\pysiglinewithargsret{\sphinxbfcode{\sphinxupquote{is\_isosceles}}}{}{}
Return whether this {\hyperref[\detokenize{api:trianglelib.shape.Triangle}]{\sphinxcrossref{\sphinxcode{\sphinxupquote{Triangle}}}}} object is isosceles.
\begin{quote}\begin{description}
\item[{Returns}] \leavevmode
whether the {\hyperref[\detokenize{api:trianglelib.shape.Triangle}]{\sphinxcrossref{\sphinxcode{\sphinxupquote{Triangle}}}}} object is isosceles

\item[{Return type}] \leavevmode
\sphinxcode{\sphinxupquote{bool}}

\end{description}\end{quote}

\end{fulllineitems}

\index{perimeter() (trianglelib.shape.Triangle method)@\spxentry{perimeter()}\spxextra{trianglelib.shape.Triangle method}}

\begin{fulllineitems}
\phantomsection\label{\detokenize{api:trianglelib.shape.Triangle.perimeter}}\pysiglinewithargsret{\sphinxbfcode{\sphinxupquote{perimeter}}}{}{}
Return the perimeter of this {\hyperref[\detokenize{api:trianglelib.shape.Triangle}]{\sphinxcrossref{\sphinxcode{\sphinxupquote{Triangle}}}}} object.
\begin{quote}\begin{description}
\item[{Returns}] \leavevmode
the perimeter of the {\hyperref[\detokenize{api:trianglelib.shape.Triangle}]{\sphinxcrossref{\sphinxcode{\sphinxupquote{Triangle}}}}} object.

\item[{Return type}] \leavevmode
\sphinxcode{\sphinxupquote{float}}

\end{description}\end{quote}

\end{fulllineitems}

\index{area() (trianglelib.shape.Triangle method)@\spxentry{area()}\spxextra{trianglelib.shape.Triangle method}}

\begin{fulllineitems}
\phantomsection\label{\detokenize{api:trianglelib.shape.Triangle.area}}\pysiglinewithargsret{\sphinxbfcode{\sphinxupquote{area}}}{}{}
Return the area of this {\hyperref[\detokenize{api:trianglelib.shape.Triangle}]{\sphinxcrossref{\sphinxcode{\sphinxupquote{Triangle}}}}} object.
\begin{quote}\begin{description}
\item[{Returns}] \leavevmode
the area of the {\hyperref[\detokenize{api:trianglelib.shape.Triangle}]{\sphinxcrossref{\sphinxcode{\sphinxupquote{Triangle}}}}} object.

\item[{Return type}] \leavevmode
\sphinxcode{\sphinxupquote{float}}

\end{description}\end{quote}

\end{fulllineitems}

\index{scale() (trianglelib.shape.Triangle method)@\spxentry{scale()}\spxextra{trianglelib.shape.Triangle method}}

\begin{fulllineitems}
\phantomsection\label{\detokenize{api:trianglelib.shape.Triangle.scale}}\pysiglinewithargsret{\sphinxbfcode{\sphinxupquote{scale}}}{\emph{\DUrole{n}{factor}}}{}
Return a new {\hyperref[\detokenize{api:trianglelib.shape.Triangle}]{\sphinxcrossref{\sphinxcode{\sphinxupquote{Triangle}}}}} object, \sphinxtitleref{factor} times the size of this one.
\begin{quote}\begin{description}
\item[{Parameters}] \leavevmode
\sphinxstyleliteralstrong{\sphinxupquote{factor}} (\sphinxcode{\sphinxupquote{float}}) \textendash{} scaling factor

\item[{Returns}] \leavevmode
a scaled new {\hyperref[\detokenize{api:trianglelib.shape.Triangle}]{\sphinxcrossref{\sphinxcode{\sphinxupquote{Triangle}}}}} object

\item[{Return type}] \leavevmode
{\hyperref[\detokenize{api:trianglelib.shape.Triangle}]{\sphinxcrossref{\sphinxcode{\sphinxupquote{Triangle}}}}}

\end{description}\end{quote}

\end{fulllineitems}


\end{fulllineitems}



\section{The “utils” module}
\label{\detokenize{api:module-trianglelib.utils}}\label{\detokenize{api:the-utils-module}}\index{module@\spxentry{module}!trianglelib.utils@\spxentry{trianglelib.utils}}\index{trianglelib.utils@\spxentry{trianglelib.utils}!module@\spxentry{module}}
Routines to test triangle properties without explicit instantiation.
\index{compute\_area() (in module trianglelib.utils)@\spxentry{compute\_area()}\spxextra{in module trianglelib.utils}}

\begin{fulllineitems}
\phantomsection\label{\detokenize{api:trianglelib.utils.compute_area}}\pysiglinewithargsret{\sphinxcode{\sphinxupquote{trianglelib.utils.}}\sphinxbfcode{\sphinxupquote{compute\_area}}}{\emph{\DUrole{n}{a}}, \emph{\DUrole{n}{b}}, \emph{\DUrole{n}{c}}}{}
Return the area of the triangle with side lengths \sphinxtitleref{a}, \sphinxtitleref{b}, and \sphinxtitleref{c}.

If the three lengths provided cannot be the sides of a triangle,
then the area 0 is returned.
\begin{quote}\begin{description}
\item[{Parameters}] \leavevmode\begin{itemize}
\item {} 
\sphinxstyleliteralstrong{\sphinxupquote{a}} (\sphinxcode{\sphinxupquote{float}}) \textendash{} side length one

\item {} 
\sphinxstyleliteralstrong{\sphinxupquote{b}} (\sphinxcode{\sphinxupquote{float}}) \textendash{} side length two

\item {} 
\sphinxstyleliteralstrong{\sphinxupquote{c}} (\sphinxcode{\sphinxupquote{float}}) \textendash{} side length three

\end{itemize}

\item[{Returns}] \leavevmode
area. If the three lengths provided cannot be the sides of a triangle, then the perimeter 0 is returned.

\item[{Return type}] \leavevmode
\sphinxcode{\sphinxupquote{float}}

\end{description}\end{quote}

\end{fulllineitems}

\index{compute\_perimeter() (in module trianglelib.utils)@\spxentry{compute\_perimeter()}\spxextra{in module trianglelib.utils}}

\begin{fulllineitems}
\phantomsection\label{\detokenize{api:trianglelib.utils.compute_perimeter}}\pysiglinewithargsret{\sphinxcode{\sphinxupquote{trianglelib.utils.}}\sphinxbfcode{\sphinxupquote{compute\_perimeter}}}{\emph{\DUrole{n}{a}}, \emph{\DUrole{n}{b}}, \emph{\DUrole{n}{c}}}{}
Return the perimeter of the triangle with side lengths \sphinxtitleref{a}, \sphinxtitleref{b}, and \sphinxtitleref{c}.

If the three lengths provided cannot be the sides of a triangle,
then the perimeter 0 is returned.
\begin{quote}\begin{description}
\item[{Parameters}] \leavevmode\begin{itemize}
\item {} 
\sphinxstyleliteralstrong{\sphinxupquote{a}} (\sphinxcode{\sphinxupquote{float}}) \textendash{} side length one

\item {} 
\sphinxstyleliteralstrong{\sphinxupquote{b}} (\sphinxcode{\sphinxupquote{float}}) \textendash{} side length two

\item {} 
\sphinxstyleliteralstrong{\sphinxupquote{c}} (\sphinxcode{\sphinxupquote{float}}) \textendash{} side length three

\end{itemize}

\item[{Returns}] \leavevmode
perimeter. If the three lengths provided cannot be the sides of a triangle, then the perimeter 0 is returned.

\item[{Return type}] \leavevmode
\sphinxcode{\sphinxupquote{float}}

\end{description}\end{quote}

\end{fulllineitems}

\index{is\_equilateral() (in module trianglelib.utils)@\spxentry{is\_equilateral()}\spxextra{in module trianglelib.utils}}

\begin{fulllineitems}
\phantomsection\label{\detokenize{api:trianglelib.utils.is_equilateral}}\pysiglinewithargsret{\sphinxcode{\sphinxupquote{trianglelib.utils.}}\sphinxbfcode{\sphinxupquote{is\_equilateral}}}{\emph{\DUrole{n}{a}}, \emph{\DUrole{n}{b}}, \emph{\DUrole{n}{c}}}{}
Return whether lengths \sphinxtitleref{a}, \sphinxtitleref{b}, and \sphinxtitleref{c} are an equilateral triangle.
\begin{quote}\begin{description}
\item[{Parameters}] \leavevmode\begin{itemize}
\item {} 
\sphinxstyleliteralstrong{\sphinxupquote{a}} (\sphinxcode{\sphinxupquote{float}}) \textendash{} side length one

\item {} 
\sphinxstyleliteralstrong{\sphinxupquote{b}} (\sphinxcode{\sphinxupquote{float}}) \textendash{} side length two

\item {} 
\sphinxstyleliteralstrong{\sphinxupquote{c}} (\sphinxcode{\sphinxupquote{float}}) \textendash{} side length three

\end{itemize}

\item[{Returns}] \leavevmode
whether lengths \sphinxtitleref{a}, \sphinxtitleref{b}, and \sphinxtitleref{c} are an equilateral triangle

\item[{Return type}] \leavevmode
\sphinxcode{\sphinxupquote{bool}}

\end{description}\end{quote}

\end{fulllineitems}

\index{is\_isosceles() (in module trianglelib.utils)@\spxentry{is\_isosceles()}\spxextra{in module trianglelib.utils}}

\begin{fulllineitems}
\phantomsection\label{\detokenize{api:trianglelib.utils.is_isosceles}}\pysiglinewithargsret{\sphinxcode{\sphinxupquote{trianglelib.utils.}}\sphinxbfcode{\sphinxupquote{is\_isosceles}}}{\emph{\DUrole{n}{a}}, \emph{\DUrole{n}{b}}, \emph{\DUrole{n}{c}}}{}
Return whether lengths \sphinxtitleref{a}, \sphinxtitleref{b}, and \sphinxtitleref{c} are an isosceles triangle.
\begin{quote}\begin{description}
\item[{Parameters}] \leavevmode\begin{itemize}
\item {} 
\sphinxstyleliteralstrong{\sphinxupquote{a}} (\sphinxcode{\sphinxupquote{float}}) \textendash{} side length one

\item {} 
\sphinxstyleliteralstrong{\sphinxupquote{b}} (\sphinxcode{\sphinxupquote{float}}) \textendash{} side length two

\item {} 
\sphinxstyleliteralstrong{\sphinxupquote{c}} (\sphinxcode{\sphinxupquote{float}}) \textendash{} side length three

\end{itemize}

\item[{Returns}] \leavevmode
whether lengths \sphinxtitleref{a}, \sphinxtitleref{b}, and \sphinxtitleref{c} are an isosceles triangle

\item[{Return type}] \leavevmode
\sphinxcode{\sphinxupquote{bool}}

\end{description}\end{quote}

\end{fulllineitems}

\index{is\_triangle() (in module trianglelib.utils)@\spxentry{is\_triangle()}\spxextra{in module trianglelib.utils}}

\begin{fulllineitems}
\phantomsection\label{\detokenize{api:trianglelib.utils.is_triangle}}\pysiglinewithargsret{\sphinxcode{\sphinxupquote{trianglelib.utils.}}\sphinxbfcode{\sphinxupquote{is\_triangle}}}{\emph{\DUrole{n}{a}}, \emph{\DUrole{n}{b}}, \emph{\DUrole{n}{c}}}{}
Return whether lengths \sphinxtitleref{a}, \sphinxtitleref{b}, \sphinxtitleref{c} can be the sides of a triangle.
\begin{quote}\begin{description}
\item[{Parameters}] \leavevmode\begin{itemize}
\item {} 
\sphinxstyleliteralstrong{\sphinxupquote{a}} (\sphinxcode{\sphinxupquote{float}}) \textendash{} side length one

\item {} 
\sphinxstyleliteralstrong{\sphinxupquote{b}} (\sphinxcode{\sphinxupquote{float}}) \textendash{} side length two

\item {} 
\sphinxstyleliteralstrong{\sphinxupquote{c}} (\sphinxcode{\sphinxupquote{float}}) \textendash{} side length three

\end{itemize}

\item[{Returns}] \leavevmode
whether lengths \sphinxtitleref{a}, \sphinxtitleref{b}, \sphinxtitleref{c} can be the sides of a triangle

\item[{Return type}] \leavevmode
\sphinxcode{\sphinxupquote{bool}}

\end{description}\end{quote}

\end{fulllineitems}



\chapter{Indices and tables}
\label{\detokenize{index:indices-and-tables}}\begin{itemize}
\item {} 
\DUrole{xref,std,std-ref}{genindex}

\item {} 
\DUrole{xref,std,std-ref}{modindex}

\item {} 
\DUrole{xref,std,std-ref}{search}

\end{itemize}


\renewcommand{\indexname}{Python Module Index}
\begin{sphinxtheindex}
\let\bigletter\sphinxstyleindexlettergroup
\bigletter{t}
\item\relax\sphinxstyleindexentry{trianglelib}\sphinxstyleindexpageref{api:\detokenize{module-trianglelib}}
\item\relax\sphinxstyleindexentry{trianglelib.shape}\sphinxstyleindexpageref{api:\detokenize{module-trianglelib.shape}}
\item\relax\sphinxstyleindexentry{trianglelib.utils}\sphinxstyleindexpageref{api:\detokenize{module-trianglelib.utils}}
\end{sphinxtheindex}

\renewcommand{\indexname}{Index}
\printindex
\end{document}